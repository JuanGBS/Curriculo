\documentclass[a4paper, 11pt]{article}

% Configuração de idioma
\usepackage[utf8]{inputenc}
\usepackage[T1]{fontenc}
\usepackage[brazil]{babel}

% Configuração de margens
\usepackage[left=2cm, right=2cm, top=2cm, bottom=2cm]{geometry}

% Ícones e Links
\usepackage{fontawesome5}
\usepackage[hidelinks]{hyperref}

% Configuração de listas e títulos
\usepackage{enumitem}
\usepackage{titlesec}

% --- Configurações Gerais ---

% Remove indentação
\setlength{\parindent}{0pt}

% Estilo dos títulos das seções
\titleformat{\section}{\Large\bfseries}{}{0em}{}[\titlerule]

% Estilo das listas (bolinha vazia)
\setlist[itemize]{label=$\circ$, left=0pt, nosep, topsep=5pt, after=\vspace{5pt}}

\begin{document}

% Cabeçalho
\begin{center}
    {\Huge \textsc{Juan Gabriel Borges Santos}} \\[0.5cm]
    
    Mr 5, Lt. 13A, Cd Marissol -- Brasília -- DF \\
    \faMobile* \ +55 61 99659-3684 \quad \textbullet \quad \faEnvelope \ \href{mailto:juangabrielborgessantos@gmail.com}{juangabrielborgessantos@gmail.com} \\
    \faLinkedin \ \href{https://www.linkedin.com/in/DevJuanSantos}{juangabrielborgessantos} \quad \textbullet \quad \faGithub \ \href{https://github.com/JuanGBS}{JuanGBS}
\end{center}

\vspace{0.3cm}

% Resumo Profissional
Atuei como Estagiário Desenvolvedor Full Stack na Caixa Econômica Federal, com experiência em desenvolvimento de interfaces web e mobile utilizando React.js e React Native, e construção de APIs RESTful com .NET (C\#). Formado em Análise e Desenvolvimento de Sistemas, com foco em boas práticas, código limpo, trabalho em equipe e aprendizado continuo, buscando consolidar carreira como desenvolvedor.

\vspace{0.3cm}

% Experiência Profissional
\section*{Experiência Profissional}

\textbf{Caixa Econômica Federal (CEF)} \hfill \textbf{Brasília, DF} \\
\textit{Estagiário Desenvolvedor Full Stack} \hfill \textit{junho 2025 -- dezembro 2025}

\begin{itemize}
    \item Desenvolvimento e manutenção de interfaces web e mobile utilizando React.js e React Native.
    \item Participação no desenvolvimento de APIs RESTful para back-end das aplicações com .NET (C\#).
    \item Colaboração com equipes de desenvolvimento e design para traduzir requisitos de negócio em soluções técnicas funcionais.
    \item Utilização de Git e metodologias ágeis (Kanban) para gestão de código e acompanhamento de tarefas.
\end{itemize}

\vspace{0.2cm}

% Educação
\section*{Educação}

Qualificações Acadêmicas \dotfill

\vspace{0.2cm}

\textbf{Faculdade Projeção} \hfill \textbf{Brasília, DF} \\
\textit{Tecnólogo em Análise e Desenvolvimento de Sistemas} \hfill \textit{março 2023 -- dezembro 2025}

\vspace{0.3cm}

% Conhecimentos Técnicos e Pessoais
\section*{Conhecimentos Técnicos e Pessoais}

\textbf{Linguagens:} Java, JavaScript/TypeScript, Python, C\# (.NET) \\
\textbf{Tecnologias:} React.js, React Native, Angular 13, .NET Core/Framework, APIs RESTful \\
\textbf{Controle de Versão:} Git, GitHub, GitLab \\
\textbf{Metodologias:} Kanban \\
\textbf{Habilidades:} Bom relacionamento interpessoal, trabalho em equipe, conhecimento básico em rotinas administrativas \\
\textbf{Idiomas:} Inglês técnico \\
\textbf{Outros:} Resolução de problemas lógicos, foco em aprendizado continuo, código limpo e modular

\vspace{0.2cm}

% Cursos e Treinamentos
\section*{Cursos e Treinamentos}

Curso de Python 3 do básico ao avançado - com projetos reais (Udemy, 141,5h) \\
Formação Angular 13 - O inicio criando 7 projetos (Udemy, 11h) \\
React practice course: Build React app from scratch (Udemy, 4,5h) \\
Front-end - HTML, CSS, JavaScript, React e + (Udemy, 51,5h) \\
Capacitação em Governança de Dados (Enap, 25h)

\end{document}